% !TEX root = ./../tesis-usach.tex
% !TEX program = xelatex
% !BIB program = bibtex
\chapter{Cálculo del gradiente de la función objetivo}
\label{apendice:dphi}
\begin{equation}
\Phi = \chi^2 + \lambda S
\end{equation}
con
\begin{eqnarray}
\chi^2 & = & \frac{1}{2}\sum^{Z}_{j=1}{\frac{|V^{m}_{j} - V_{j}^{o}|^{2}}{\sigma_j^{2}}} \\
S & = & \sum^{n}_{k=1}{I_{k}\log{(I_{k}/G)}}
\end{eqnarray}
con $0 \leq \lambda < \infty$.

Calculando el gradiente de la función $\Phi$, es decir, derivando parcialmente con respecto a $I_{k}$:

\begin{equation}
[\nabla\Phi]_k = \frac{\partial \Phi}{\partial I_{k}} = \frac{\partial}{\partial I_{k}} \chi^{2} + \lambda \frac{\partial}{\partial I_{k}}S
\label{eq:fina}
\end{equation}



Primero derivamos la derivada de $\chi^2$.
\begin{eqnarray}
\frac{\partial\chi^{2}}{\partial I_{k}} & = & \frac{\partial}{\partial I_{k}}\frac{1}{2}\sum\limits_{j}\frac{(V^{m}_{j} - V_{j}^{o})^{*}\cdot(V^{m}_{j} - V_{j}^{o})}{\sigma_{j}^{2}}\\
                                        & = & \frac{1}{2}\sum\limits_{j}\frac{1}{\sigma_{j}^{2}}\bigg(\frac{\partial
                                         V^{m*}_{j}}{\partial I_{k}}\Bigl( V_{j}^{m} - V_{j}^{o}\Bigr)+\frac{\partial V_{j}^{m}}{\partial I_{k}}\Bigl(V_{j}^{m}- V_{j}^{o}\Bigr)^{*}\biggr)
\end{eqnarray}

Note que $V_j^{o}$ es constante y por lo tanto su derivada respecto a $I_k$ es cero.
Donde la transformada de Fourier de la imagen ($I$) es igual a :

\begin{equation}
V_{j}^{m}=\sum^{n}_{k=1}I_{k}e^{-2\pi i<X_k,Z_j>}
\end{equation}

Donde $Z_j=(u_j,v_j)$, $X_k=(x_k,y_k)$, y el producto interno $<(a,b),(c,d)>=ac+bd$. Luego el conjugado es:

\begin{equation}
V_{j}^{m*}=\sum^{n}_{k=1}I_{k}e^{2\pi i<X_k,Z_j>}
\end{equation}

Derivando ambas visibilidades por separado con respecto a $I_{k}$ se tiene que:

\begin{eqnarray}
\frac{\partial V_{j}}{\partial I_{k}} & = & \exp\{-2\pi i<X_k,Z_j>\} \\
\frac{\partial V_{j}^{*}}{\partial I_{k}} & = & \exp\{+2\pi i<X_k,Z_j>\}
\end{eqnarray}

Definiendo el residuo de las visibilidades como:

\begin{equation}
V_{j}^{R} = V_{j}^{m}-V_{j}^{o} = \text{Re}(V_{j}^{R}) + i\cdot \text{Im}(V_{j}^{R})
\end{equation}

Y

\begin{equation}
V_{j}^{R*} = (V_{j}^{m}-V_{j}^{o})^{*}=\text{Re}(V_{j}^{R})-i\cdot \text{Im}(V_{j}^{R})
\end{equation}

Reemplazando estos términos se tiene:


\begin{equation}
\frac{\partial\chi^{2}}{\partial I_{k}} = \frac{1}{2}\sum\limits_{j}\frac{1}{\sigma_{j}^{2}}\Bigl[\exp\{2\pi i<X_k,Z_j>\}\Bigl(\text{Re}(V_{j}^{R})+i\text{Im}(V_{j}^{R})\Bigr)+
              \exp\{-2\pi i<X_k,Z_j>\}\Bigl(\text{Re}(V_{j}^{R})-i\text{Im}(V_{j}^{R})\Bigr)\Bigr]
\end{equation}

Multiplicando distributivamente:
%\begin{equation}
\begin{multline}
\frac{\partial\chi^{2}}{\partial I_{k}} = \frac{1}{2}\sum\limits_{j}\frac{1}{\sigma_{j}^{2}}\biggl[\exp\{2\pi i<X_k,Z_j>\}\text{Re}(V_{j}^{R})+i\exp\{2\pi i<X_k,Z_j>\}\text{Im}(V_{j}^{R})-\\
\exp\{-2\pi i \langle X_k,Z_j \rangle\}\text{Re}(V_{j}^{R})-i\exp\{-2\pi i\langle X_k,Z_j \rangle\}\text{Im}(V_{j}^{R})\biggr]
\end{multline}
%\end{equation}


Factorizando por la partes reales e imaginarias de $V_{j}^{R}$:

\begin{multline}
\frac{\partial\chi^{2}}{\partial I_{k}} = \frac{1}{2}\sum\limits_{j}\frac{1}{\sigma_{j}^{2}}\biggl[\text{Re}(V_{j}^{R})\Bigl(\exp\{2\pi i<X_k,Z_j>\}+\exp\{-2\pi i<X_k,Z_j>\}\Bigr)+ \\ i\text{Im}(V_{j}^{R})\Bigl(\exp\{2\pi i<X_k,Z_j>\}-\exp\{-2\pi i<X_k,Z_j>\}\Bigr)\biggr]
\end{multline}

Usando las propiedades trigonométricas del seno y coseno se tiene que:

\begin{eqnarray}
\frac{\partial\chi^{2}}{\partial I_{k}} & = & \frac{1}{2}\sum\limits_{j}\frac{1}{\sigma_{j}^{2}}\biggl[2\text{Re}(V_{j}^{R})\cos(2\pi <X_k,Z_j>)+2i^{2}\text{Im}(V_{j}^{R})\sin(2\pi \langle X_k,Z_j\rangle )\biggr] \\
                                      & = & \sum\limits_{j}\frac{1}{\sigma_{j}^{2}}\biggl[\text{Re}(V_{j}^{R}) \cos(2\pi\langle X_k,Z_j\rangle) - \text{Im}(V_{j}^{R})\sin(2\pi \langle X_k,Z_j\rangle)\biggr]
\end{eqnarray}

Definiendo

\begin{equation}
W_{j} \propto \frac{1}{\sigma_{j}^{2}}
\end{equation}

tenemos finalmente que

\begin{equation}
\frac{\partial\chi^{2}}{\partial I_{k}} = \sum\limits_{j}W_{j}\biggl[Re(V_{j}^{R})\cos\bigl(2\pi \langle X_k,Z_j\rangle\bigr)-\text{Im}(V_{j}^{R})\sin\bigl(2\pi \langle X_k,Z_j\rangle\bigr)\biggr]
\label{eq:dchi2}
\end{equation}

Continuando con $\frac{\partial}{\partial I_{k}}S$

\begin{equation}
\frac{\partial S}{\partial I_{k}} = \frac{\partial}{\partial I_{k}}\sum\limits_{k}I_{k}\log(I_{k}/G)
\end{equation}

Esto es igual a:


\begin{equation}
\frac{\partial S}{\partial I_{k}} = 1+\log(I_{k}/G)
\end{equation}


Por tanto la coordenada $k$ de $\nabla\Phi$:

\begin{multline}
[\nabla \Phi]_k = \sum\limits_{j}W_{j}\biggl[\text{Re}(V_{j}^{R})\cos\bigl(2\pi \langle X_k,Z_j \rangle\bigr)-\text{Im}(V_{j}^{R})\sin\bigl(2\pi \langle X_k,Z_j \rangle\bigr)\biggr] +
\lambda(1 + \log(I_{k}/G))
\label{eq:dphifinal}
\end{multline}
